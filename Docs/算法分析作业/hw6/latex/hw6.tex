\documentclass[a4paper,10pt]{article}
\usepackage{ctex}
\usepackage{fancyhdr}
\usepackage{mflogo,texnames}
\usepackage[latin1]{inputenc}
\usepackage{graphicx}
\usepackage{amssymb}
\usepackage{amsmath}
\usepackage{epstopdf} 
\usepackage{listings}
% 插入伪代码
\usepackage[linesnumbered,boxed]{algorithm2e}
\usepackage{color}
\definecolor{keywordcolor}{rgb}{0.8,0.1,0.5}
\definecolor{webgreen}{rgb}{0,.5,0}
\usepackage{geometry}
\geometry{left=3.18cm,right=3.18cm,
    top=2.54cm,bottom=2.54cm,
    head=1.5cm,foot=1.75cm}
\usepackage{multirow}

\pagestyle{fancy}
\lhead{学号:1401214342}
\chead{姓名: 韩喆}
\rhead{}
\lfoot{Han Zhe(icst@pku)}
\cfoot{iampkuhz@gmail.com}
\rfoot{\thepage}

% word page layout
%\usepackage[top=2.54cm, bottom=2.54cm, left=3.18cm, right=3.18cm]{geometry}

\begin{document}
\begin{center}
\LARGE 算法课第 6次作业
\end{center}

\begin{center}
 \begin{tabular}{|c|c|c|c|c|}
\hline
      & 题目1 & 题目2 & 题目3 & 总分\\ \hline
 \multirow{2}{*}{分数} &\multirow{2}{*}{} &\multirow{2}{*}{} &\multirow{2}{*}{} &\multirow{2}{*}{}\\
 & & & & \\ \hline
 \multirow{2}{*}{阅卷人} &\multirow{2}{*}{} &\multirow{2}{*}{} &\multirow{2}{*}{} &\multirow{2}{*}{}\\
 & & & &  \\ \hline
\end{tabular}
\end{center}
\vspace{20pt}
\section{最小割不变}
  \normalsize
  \subsection{为假} 
  \paragraph{反例} 假设图中点集为$\{s, v_1, v_2, w, t\}$,边集为$\{(s,v_1,2),(s,v_2,2),(v_1,w,2),(v_2,w,2),(w,t,4)\}$(a,b,c)表示从a到b的边且容量为c,A为点s,B其余点以及其内部的边。则(A,B)是一个最小割,且大小为4,而提升之后的A,B割的容量为6,但是最小割为({t},其余),容量为5,所以更新容量之后,不再是最小割。
  \vspace{100pt}
\section{唯前向边算法}
  \subsection{(1)} 错误
  \paragraph{}反例
  点集为$\{s, v_1,...,v_{b+1}, u_1,...,u_{b+1}, t\}$,边集为
  
                                                     $\{(s,u_1),...,(s,u_{b+1}),(u_1,v_1),...,(u_{b+1},v_{b+1})$,
                                                     $(v_1,t),...,(v_{b+1},t),(u_1,v_2),...,(u_b,v_{b+1})\}$
  所有边的容量均为1
  如果第一次找到的增广路径是$s->u_1->v_1->u_2->v_2->u_3->...->u_b-v_b->u_{b+1}->v_{b+1}->t$,则之后在唯向前边的剩余图中不存在增广路径,所以此时计算的最大流为1;但实际上最大流为b+1,为b+1条$s->u_{b+1}->v_{b+1}->t$
  \newpage
  \vspace{100pt}
  \subsection{(2)} 错误
  \paragraph{}反例
  点集为$\{s, v_1,...,v_{b+1}, u_1,...,u_{b+1}, t\}$,边集为
                                                     $\{(s,u_1),...,(s,u_{b+1}),(u_1,v_1),...,(u_{b+1},v_{b+1})$,
                                                     $(v_1,t),...,(v_{b+1},t),(v_1,u_2),...,(v_1,u_{b+1})\}$
  其中$(s,u_2),...,(s,u_{b+1}),(v_1,t)$的容量为1,$(s,u_1),(u_1,v_1)$容量为bm,其余容量为m
  使用BFS查找到的最大流为b+1(b+1条路径$s->u_i->v_i->t$)但是实际上最大流流量大于bm($s->u_1$充满,分成b份在$v_1$分发个$u_2,...,u_{b+1}$,再由$v_2,...,v_{b+1}$流向t)。所以此时的流量小于最大流的$\frac{b+1}{bm}$,取m=(b+1)(b+1),为$\frac{1}{(b+1)b}<\frac{1}{b}$
\end{document}
