\documentclass[xcolor=svgnames,dvipsnames,table, hyperref=pdftex, mathserif, presentation]{beamer}
\usepackage{amsmath,amssymb,amsfonts,amsthm}
\usepackage{ctex}
\usepackage{graphics}
\usepackage{graphicx}
\usepackage{xcolor}
\usepackage{wasysym}
\usepackage{bbm}
\usepackage{url}
\usepackage{beamerleanprogress}
\usepackage{tikz-dependency}
\usepackage{tikz-qtree}

\usetheme{CambridgeUS}
%\usetheme{Pittsburgh}
\usecolortheme{orchid} % seahorse  orchid rose
\setbeamertemplate{blocks}[rounded][shadow=true]
\AtBeginSection[]{%
  \begin{frame}<beamer>
    \frametitle{Outline}
      \tableofcontents[current] 
    \end{frame}
  \addtocounter{framenumber}{-1}% If you don't want them to affect the slide number
}
\AtBeginSubsection[]
{
  \begin{frame}
  \frametitle{Outline}
    \tableofcontents[currentsection,currentsubsection]
  %\tableofcontents[sectionstyle=show/hide,subsectionstyle=hide/show/hide]
  \end{frame}
  \addtocounter{framenumber}{-1}% If you don't want them to affect the slide number
}
\newcommand{\setof}[1]{\ensuremath{\left \{ #1 \right \}}}
\newcommand{\tuple}[1]{\ensuremath{\left \langle #1 \right \rangle }}
\newcommand{\red}[1]{\textcolor{red}{#1}}
\newcommand{\brown}[1]{\textcolor{brown}{#1}}
\newcommand{\green}[1]{\textcolor{green}{#1}}
\newcommand{\blue}[1]{\textcolor{blue}{#1}}
\newcommand{\cyan}[1]{\textcolor{cyan}{#1}}

%gets rid of navigation symbols
\setbeamertemplate{navigation symbols}{}

\begin{document}

\title[SCKR 2014]{Quality Control for \textbf{Structured} Knowledge Base}

\institute[icst@pku]{
  
}
\author[Zhe Han]{\\ Zhe Han \\ 1401214342\\ iampkuhz@gmail.com
}

\frame[t,plain]{ \titlepage } % [t,plain]

\frame{
  \frametitle{ Outline  }
  
   \begin{itemize}
  \item Quality control challenge in structured knowledge base
    \begin{itemize}
     \item definition
     \item challenge
    \end{itemize}

  \item QC in/before generatings part
    \begin{itemize}
      \item YAGO2s: extractor, type checker 
      \item YAGO3: multilingual resources
      \item tendency: semi-structured KB, restriants, multilingual resources,
    \end{itemize}

  \item KB repair and enrichment
    \begin{itemize}
    \item Freebase: predict contribution quality
    \end{itemize}
  \item Summary
  \end{itemize}

}

\frame{
  \frametitle{structured knowledge base}
  \begin{itemize}
   \item Triples: <S, P, O> 
   \begin{itemize}
    \item form of subject-predicate-object
   \end{itemize}

   \item Schema: 
   \begin{itemize}
    \item predicate/relation/property set: definition(restriction)
    \item recognition of domain and domain type...
   \end{itemize}
   \item Taxonomy:
   \begin{itemize}
    \item (domain's classses)
    \item class hierarchy
   \end{itemize}
   
  \end{itemize}

}

\frame{
  \frametitle{ Challenge }
  
    \begin{itemize}
      \item outdate/duplicate entries
      \item unification of predicate, constraint of relation
      \item inconsistent formats/data\textbf{\red{(no commercial KB available)}}
      \item disambiguation names, entity linking
      \item ...
    
    \end{itemize} 

}

\frame{
  \frametitle{QC in/before generation part}
  
  \subtitle{middle}{
  \begin{center}\Large YAGO2s: extractor\\ \end{center}
  \begin{block}{paper}
   (www2013)Inside YAGO2s: A Transparent Information Extraction Architecture\\
   (btw2013)YaAGO2s: Modular High-Quality Information Extraction with an Application to Flight Planning
  \end{block}

  }                                                 
}

\frame{
  \frametitle{YAGO2s: extractor}
  \begin{itemize}
   \item Motivation of YAGO2s:
   \begin{itemize}
    \item Achieving the data quality with more people joining.
   \end{itemize}
   
   \item Features of YAGO2s:
    \begin{itemize}
     \item Modularizd IE process
     \item Modulazied data(by theme)
     \item parallelizd and Distributed, thus fast
    \end{itemize}
   \item Mothod/Architecture
    \begin{itemize}
     \item 39/46 extractor modules:\begin{tiny}
                                    duplicate facts, check types, constraints...
                                   \end{tiny}
     \item interact extrctors for checking:
     \begin{itemize}
      \item Extractors to display input/output/proc/provenance of other extrctors
      \item Modifying extractors to modify input of other extractors
     \end{itemize}
    \end{itemize}

  
  \end{itemize}

}
\frame{
  \frametitle{YAGO2s: extractor}
  \begin{itemize}
   \item YAGO2s architexture example:\begin{tiny}
                                       theme: a set of facts
                                     \end{tiny}
  \begin{figure}[h]
  \centering
  \includegraphics[width=0.5\textwidth]{yago2sArchitecture}
  \end{figure}
  \item Extractor
  \begin{itemize}
   \item input: theme(output of other extractor)/data source(Wikipedia, WordNet)
   \item output: one or more themes
  \end{itemize}

  \end{itemize}

}

\frame{
  \frametitle{QC in/before generation part}
  
  \subtitle{middle}{
  \begin{center}
  \Large {YAGO3: a KB from Multilingual Wikipedias}\\
  \tiny{Although it is just for expanding the KB, its method can also be applyed to control quality}\\
  \end{center}
  \begin{block}{paper}
   (CIDR 2015) 
   YAGO3: A Knowledge Base from Multilingual Wikipedias
  \end{block}

  }                                                 
}

%\frame{
%  \frametitle{KB repair and enrichment}
%  \subtitle{middle}{
%  \begin{center}YAGO3: a KB from Multilingual Wikipedias\end{center}
%  }                                                 
%}


\frame{
  \frametitle{YAGO3: a KB from Multilingual Wikipedias}
  \begin{itemize}
   \item resources and steps
   \begin{itemize}
    \item wikipedia-> infoboxes, classes; wikidata-> entities
    \item fact(relation) extraction
      \begin{itemize}
       \item extract from infoboxes: attributes and values;
       \item type checking: <person, bornIn, place>
       \item function checking: <a, bornIn, p1>, <a, bornIn, p2> -> p1=p2
       \item ...
       \item finally map to 77-English-YAGO relations
      \end{itemize}
    \item cross-linguistic mapping(co-occurence)
      \begin{itemize}
       \item infoboxes mapping: entity/domain mapping
       \item attribute mapping:more common, less different, frequancy...
      \end{itemize}
    \item taxonomy extraction
      \begin{itemize}
        \item category extractor
	\item English class hierarchy by WordNet 
	\item Multilugal class hierarchy mapped by langlink resources
      \end{itemize}

   \end{itemize}

  \end{itemize}

}

\frame{
  \frametitle{YAGO3: a KB from Multilingual Wikipedias}
  \begin{itemize}
   \item on Quality control...
   \begin{itemize}
    \item small KB (like Arbric) can expand more features while training
    \item estimate possible insconsistency triples by other language data
    \item ...
   \end{itemize}

  \end{itemize}

}

\frame{
  \frametitle{YAGO3: a KB from Multilingual Wikipedias}
  \begin{itemize}
   \item Yaoming at en.wikipedia.org
  \begin{figure}[h]
  \centering
  \includegraphics[width=0.9\textwidth]{yaoming}
  \end{figure}
   
  \end{itemize}

}

\frame{
  \frametitle{KB repair and enrichment}
  \subtitle{middle}{
  \begin{center}\Large freebase: predicting contribution quality for KB construction\end{center}
  \begin{block}{paper}
   (wsdm2014)Trust, but verify: predicting contribution quality for knowledge base construction and curation
  \end{block}

  }                                                 
}

\frame{
  \frametitle{freebase: predicting contribution quality for KB construction}
  \begin{itemize}
   \item moviation
    \begin{itemize}
     \item verified a new contribution is correct immediately(\red{\textbf{real time}}) after being submitted
    \end{itemize}
  \item problem definition
    \begin{itemize}
     \item extract features(contributions and users)
     \item change the problem to supervised binary classification
    \end{itemize}
  \item process
    \begin{itemize}
     \item features
      \begin{itemize}
       \item user contribution history: action, deletion,...
       \item \textbf{user contribution expertise}:
       \item triple features: frequently revised
      \end{itemize}
    \item contribution similarity
      \begin{itemize}
       \item new contribution-> find topics, taxonomy, predicate and expertise -> previous proposed good 
      \end{itemize}
    \item \red{(distant supervised)} training
      \begin{itemize}
       \item triples in training set are labeled due to their lifetime(\red{$\geq 4 weeks$}).
      \end{itemize}

    \end{itemize}

  \end{itemize}

}

\frame{
  \frametitle{freebase:predicting contribution quality for KB construction}
  \begin{itemize}
   \item experiment conclusion
   \begin{itemize}
    \item logistic regression is better than perceptron and gradboost
    \item \textbf{user history features is of least importance}
    \item expertise info is of great importance
   \end{itemize}
   \item future improvement?
    \begin{itemize}
     \item hard constraints on schema rules
    \end{itemize}

  \end{itemize}

}

\frame{
  \frametitle{Summary}
  \begin{itemize}
   \item quality control in structured KB
    \begin{itemize}
      \item based on good resources
     \item based on frequent pattern
    \end{itemize}

  \end{itemize}
  
    \begin{block}{candidate solutons}

    \begin{itemize}
     \item High-quality resources: \begin{footnotesize}
                              unified hierarchy, diambiguated domain... \\ Wikipedia, WordNet, GeoNames...
                             \end{footnotesize}
     \item Aalingment: ontology alingment, cross-language alingment
     \item Expert modification: \begin{footnotesize}
                                 expertise estimation,...
                                \end{footnotesize}

     \item ?
    \end{itemize}
    \end{block}

}
\frame{
  \frametitle{Appendix}
  \begin{itemize}
   \item attribute mapping(co-occurence):
   \begin{itemize}
    \item the more common subj and obj, the beter:$match(F_a,F_r)= \pi_{subj}(F_a\cap F_r)$
    \item the more different, the worse:
	  $clashes(T_a,F_R)=\pi_{sub}(F_a)\cap (\pi_{sub}(F_r)\setminus\pi_{subj}(F_a\cap F_r))$
   \end{itemize}
   
   \item user contribution expertise
   \begin{itemize}
    \item 1m topics(\red{too sparse}), 5k taxonomy, 3k predicates
   \end{itemize}

  \end{itemize}

}
\end{document}
