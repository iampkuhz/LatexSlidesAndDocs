\documentclass[xcolor=svgnames,dvipsnames,table, hyperref=pdftex, mathserif, presentation]{beamer}
\usepackage{amsmath,amssymb,amsfonts,amsthm}
\usepackage{ctex}
\usepackage{graphics}
\usepackage{graphicx}
\usepackage{xcolor}
\usepackage{wasysym}
\usepackage{bbm}
\usepackage{url}
\usepackage{beamerleanprogress}
\usepackage{tikz-dependency}
\usepackage{tikz-qtree}

\usetheme{CambridgeUS}
%\usetheme{Pittsburgh}
\usecolortheme{orchid} % seahorse  orchid rose
\setbeamertemplate{blocks}[rounded][shadow=true]
\AtBeginSection[]{%
  \begin{frame}<beamer>
    \frametitle{Outline}
      \tableofcontents[current] 
    \end{frame}
  \addtocounter{framenumber}{-1}% If you don't want them to affect the slide number
}
\AtBeginSubsection[]
{
  \begin{frame}
  \frametitle{Outline}
    \tableofcontents[currentsection,currentsubsection]
  %\tableofcontents[sectionstyle=show/hide,subsectionstyle=hide/show/hide]
  \end{frame}
  \addtocounter{framenumber}{-1}% If you don't want them to affect the slide number
}
\newcommand{\setof}[1]{\ensuremath{\left \{ #1 \right \}}}
\newcommand{\tuple}[1]{\ensuremath{\left \langle #1 \right \rangle }}
\newcommand{\red}[1]{\textcolor{red}{#1}}
\newcommand{\brown}[1]{\textcolor{brown}{#1}}
\newcommand{\green}[1]{\textcolor{green}{#1}}
\newcommand{\blue}[1]{\textcolor{blue}{#1}}
\newcommand{\cyan}[1]{\textcolor{cyan}{#1}}

%gets rid of navigation symbols
%\setbeamertemplate{navigation symbols}{}

\begin{document}

\title[中文谓词归一化]{基于中文维基百科构建的\\ 知识库的谓词归一化}

\institute[icst.wip@pku]{
  
}
\author[Zhe Han]{\\ 韩喆 \\ iampkuhz@gmail.com
}

\frame[t,plain]{ \titlepage } % [t,plain]

\frame{
  \frametitle{ Outline  }
  
   \begin{itemize}
  \item 背景
    \begin{itemize}
     \item 基于维基百科的知识库
    \end{itemize}

  \item Motivation

  \item 实验方法
  
  \item 特征选取

  \item 实验效果
    \begin{itemize}
    \item 分析和改进
    \end{itemize}
  \end{itemize}

}

\frame{
  \frametitle{Background}
  基于维基百科的知识库
  \begin{itemize}
   \pause
   \item definition
      \begin{itemize}
       \item 给定一组句子, 判断其是否是复述
	  \begin{itemize}
	   \item binary classification
	  \end{itemize}

      \end{itemize}
      
    \item Microsoft Research Paraphrase Corpus (MSRP)
	\begin{itemize}
	 \item train: 4,076 sentence pairs (2,753 positive: 67.5 \%)
	 \item test: 1,725 sentence pairs (1,147 positive: 66.5 \%)
	 \item 2个标注者, 83\%的一致性, 第三个人更正
	\begin{block}{Sample data}
	 \begin{itemize}
	 	 \item Sentence 1: Amrozi accused his brother, whom he called "the witness", of deliberately distorting his evidence.
	       \item Sentence 2: Referring to him as only "the witness", Amrozi accused his brother of deliberately distorting his evidence.
	       \item Class: 1 (true paraphrase)
	 \end{itemize}
	\end{block}
	
	\end{itemize}
  \end{itemize}

}


\frame{
    \frametitle{Paraphrase identification}
    \begin{itemize}
     \item Common methods
	\begin{itemize}
	  \item lexical features
	      \begin{itemize}
	       \item n-gram features, skip-gram fatures, ...
	      \end{itemize}

	  \item semantic features
	      \begin{itemize}
	       \item POS tag, wordnet similarity, dependency tree relation, ...
	      \end{itemize}

	  \item classification
	      \begin{itemize}
	       \item SVM, voted classifications
	      \end{itemize}
	\end{itemize}
    
    \item Challenge
	\begin{itemize}
	 \item 没有提取句子的全局信息( dependency features利用不足)
	 \item 对句子涵义的特征提取不足(没有真正理解句子)
	\end{itemize}

    \end{itemize}

}

\end{document}
