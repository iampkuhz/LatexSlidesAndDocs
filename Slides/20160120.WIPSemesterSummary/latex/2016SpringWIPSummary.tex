

\documentclass[color=usenames,dvipsnames]{beamer}

\usepackage[adobefonts,noindent]{ctex} %中文支持
\setCJKmainfont{SimSun}

\mode<presentation> {

\usetheme{Madrid}
\usecolortheme{lily}
\useoutertheme{infolines}

}


\usepackage{booktabs} 
\usepackage{tikz}

% for gif
% \usepackage[dvipdfmx]{movie15}
\usepackage{graphicx}
\usepackage{animate}
\usepackage{media9}

% Thin fonts
\usepackage{cmbright}
\usepackage[T1]{fontenc}

\definecolor{dark_grey}{gray}{0.5}
\setbeamercolor{normal text}{fg=dark_grey,bg=white}
\setbeamertemplate{navigation symbols}{}

\setbeamercolor*{palette primary}{fg=gray!100,bg=gray!10}
\setbeamercolor*{palette quaternary}{fg=gray!100,bg=gray!10}
\setbeamercolor*{palette secondary}{fg=gray!100,bg=gray!20}
\setbeamercolor*{palette tertiary}{fg=gray!100,bg=gray!10}
\setbeamercolor*{navigation symbols}{fg=white,bg=white}
\usefonttheme{default}


\setbeamertemplate{blocks}[rounded][shadow=false]
\setbeamercolor{block title}{bg=gray!10}
\setbeamercolor{block body}{fg=gray,bg=gray!10}
%\setbeamercolor{frametitle}{fg=}

\setbeamertemplate{frametitle}[default][center]

\setbeamertemplate{itemize items}[default]
\setbeamertemplate{enumerate items}[default]

\newcommand{\F}{\mathbb{F}}


\title[2015 Summ]{2015秋季学期总结}
\author{韩喆}
\institute{PIE@WIP}
\date{2016-01-21}
\begin{document}


\begin{frame}
  \titlepage
\end{frame}

% Uncomment these lines for an automatically generated outline.
%\begin{frame}{Outline}
%  \tableofcontents
%\end{frame}

\begin{frame}


\section{Introduction}
\frametitle{工作简介}
\begin{itemize}
\item  助教:冯老师开的2015级工学院《计算概论》
\item \color{Green}
 组会报告:大组会2次+小组会2次
\item\color{RedOrange} 实验室新主页
\color{blue}
\item 结构化知识库构建+展示
\end{itemize}
\end{frame}


\begin{frame}{助教}
 \begin{block}{}
  2015级工学院计算概论
 \end{block}

 \begin{itemize}
  \item 和吕超、李友焕师兄担任助教
  \item 每周一下午的上机辅导
  \item 两次课堂答疑/习题课,期中/期末监考,阅卷
 \end{itemize}
 
 \pause
 感想
 \begin{itemize}
  \item 感觉回到了本科的时候,但是没有绩点的压力了
  \item 编程能力还有待提高
 \end{itemize}
\end{frame}

\begin{frame}{组会报告}
 \begin{block}{大组会}
  \begin{itemize}
   \item EMNLP 2015 的一篇做不同知识库的合并工作的文章
    \begin{itemize}
     \item Knowledge Base Unification via Sense Embeddings and Disambiguation
     \item 合并实体、合并谓词
    \end{itemize}
   \item 目前工作:中文知识库构建
    \begin{itemize}
     \item 构建自己的\begin{color}{blue}类别\end{color}和\begin{color}{red}谓词\end{color}集合
    \end{itemize}
  \end{itemize}
 \end{block}

 \begin{block}{小组会}
  \begin{itemize}
   \item 介绍了几个自己在用的工具/软件,
  \end{itemize}

 \end{block}

\end{frame}


\begin{frame}{Introduction}

\animategraphics[loop,height=0.5\textheight]{2}{pic/ontology-}{0}{56}
\end{frame}

\end{document} 