\documentclass[xcolor=svgnames,dvipsnames,table, hyperref=pdftex, mathserif, presentation]{beamer}
\usepackage{amsmath,amssymb,amsfonts,amsthm}
\usepackage{ctex}
\usepackage{graphics}
\usepackage{graphicx}
\usepackage{xcolor}
\usepackage{wasysym}
\usepackage{bbm}
\usepackage{url}
\usepackage{beamerleanprogress}
\usepackage{tikz-dependency}
\usepackage{tikz-qtree}
\usepackage{hhline}
\usepackage{fancyvrb}
\usepackage{mathrsfs}
\usepackage{multirow}
\usepackage{alltt}
\usepackage[subrefformat=parens]{subcaption}
% for uml charts
\usepackage{tikz}
\usetikzlibrary{calc,arrows.meta, graphs, trees, shapes, positioning, automata,
shapes.geometric, shapes.multipart, er, patterns, decorations.markings, intersections, decorations.text}
\usepackage{tikz-uml}

\usetheme{CambridgeUS}
%\usetheme{Pittsburgh}
\usecolortheme{orchid} % seahorse  orchid rose
\setbeamertemplate{blocks}[rounded][shadow=true]
\AtBeginSection[]{%
  \begin{frame}<beamer>
    \frametitle{Outline}
      \tableofcontents[current] 
    \end{frame}
  \addtocounter{framenumber}{-1}% If you don't want them to affect the slide number
}
\AtBeginSubsection[]
{
  \begin{frame}
  \frametitle{Outline}
    \tableofcontents[currentsection,currentsubsection]
  %\tableofcontents[sectionstyle=show/hide,subsectionstyle=hide/show/hide]
  \end{frame}
  \addtocounter{framenumber}{-1}% If you don't want them to affect the slide number
}
\newcommand{\setof}[1]{\ensuremath{\left \{ #1 \right \}}}
\newcommand{\tuple}[1]{\ensuremath{\left \langle #1 \right \rangle }}
\newcommand{\red}[1]{\textcolor{red}{#1}}
\newcommand{\brown}[1]{\textcolor{brown}{#1}}
\newcommand{\green}[1]{\textcolor{green}{#1}}
\newcommand{\blue}[1]{\textcolor{blue}{#1}}
\newcommand{\cyan}[1]{\textcolor{cyan}{#1}}

%gets rid of navigation symbols
%\setbeamertemplate{navigation symbols}{}

\begin{document}

\title[2017.Sum]{2016年秋季学期总结}

\institute[icst@pku]{
  Institute of Computer Science \& Technology \\
  Peking University
}
\author[Zhe]{
  Zhe, Han
}

\frame[t,plain]{ \titlepage } % [t,plain]

\frame{
  \frametitle{ Outline }
   \begin{itemize}
      \item 找工作
      \item 学术
      \item 科研 \&\& 实验室工作
   \end{itemize}
}

\frame{
  \frametitle{找工作}
  \begin{columns}
   \column{0.7\hsize}
    \begin{itemize}
     \item 杭州\blue{蚂蚁金服}
     \item 9.10 拿到 offer, 后面找工作的时候划水...
    \end{itemize}
    \end{columns}
}

% 
% \frame{
%   \frametitle{找工作}
%   \begin{columns}
%    \column{0.3\hsize}
%     \begin{itemize}
%      \item 杭州\blue{蚂蚁金服}
%      \item 9.9 拿到 offer
%      \item 后面划水
%     \end{itemize}
% 
%    \column{0.7\hsize}
%     \begin{block}{杭州实习/生活}
%      \begin{enumerate}\small
%       \item \red{IOT}工作室,对话机器人+AR/VR
%       \item 工作内容:\sout{机器人项目维护}+AR安卓demo
%       \item 团队基本NLP出身, 人nice, 有牛人
%       \item 频繁的对外交流:码力vs合作?新项目成就感vs弯路?
%       \item 双休,9点半-10点上班,7-9点下班?
% 	\begin{itemize}\footnotesize
% 	 \item 组里很多人家人不在杭州,周末回家
% 	\end{itemize}
%       \item 一个人的生活单调,奥运会/跑步
%       \item 杭州潮湿
%       \item 感谢沈许川/饶俊阳师兄
%      \end{enumerate}
%     \end{block}
% 
%   \end{columns}
% }


\frame{
  \frametitle{学术}
  \begin{itemize}
   \item 终于发了文章
    \begin{itemize}
     \item Detecting Synonymous Predicates from Online Encyclopedia with Rich Features
     \item AIRS 2016 长文
     \item 11.24 骑ofo去清华做了报告
    \end{itemize}
   \item 一直拖着的知识库构建的期刊...
  \end{itemize}
}

\frame{
  \frametitle{科研 \&\& 实验室工作}
    \begin{itemize}
     \item 知识库
      \begin{itemize}
       \item 完成了\blue{百度百科}数据爬取、整理
	\begin{itemize}
	 \item 各种坑:页面存储、磁盘空间、页面去重、别名表抽取与排序、PKUBase类别对应
	 \item 1100w实体,4500w三元组,1800w的类别信息,1500w的别名表
	\end{itemize}
       \item 找到了\textbf{接班人}...
	\begin{itemize}
	 \item 带吴雨婷师妹做了一些东西(任务很艰巨),希望把知识库的维护做下去
	\end{itemize}
       \item 做了\red{新版的前端}展示界面
	\begin{itemize}
	 \item 自认为界面友好度or美观程度提高了不少...
	\end{itemize}
       \item 方正提供了一个知识库版本
      \end{itemize}
     \item 新华社新闻分类
      \begin{itemize}
       \item 回来后整理了一下之前的数据
       \item 帮老师跑了几次会、交了一些材料,终于算是结项了
      \end{itemize}
    \end{itemize}

}

\frame{
  \frametitle{科研 \&\& 实验室工作}
    \begin{itemize}
     \item 下一步工作
      \begin{itemize}
       \item \brown{进行中}:知识库\red{实体链接}
       \item \brown{进行中}:百度百科与维基百科的融合
	\begin{itemize}
	 \item 目前2个知识库版本已经有了各自的别名表、三元组、PKUBase类别信息
	\end{itemize}
       \item \red{期刊}...
      \end{itemize}
    \end{itemize}
}

\frame{
  \begin{columns}[c]
   \column{.3\hsize}
   \column{.4\hsize}
   \begin{block}{}
    \centering \Large 谢谢大家! \\ 
   \end{block}
   \column{.3\hsize}
  \end{columns}
}

\end{document}